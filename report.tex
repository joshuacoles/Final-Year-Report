\documentclass[10pt]{iopart}
% \usepackage{setstack}
\usepackage{iopams}

\usepackage{physics}

\usepackage{todonotes}
\usepackage{cite}
\usepackage{lipsum}
\usepackage{hyperref}
\usepackage{csquotes}

\newcommand{\PHYSICS}{\textbf{PHYSICS!!!!}
}
\newcommand{\R}{\mathbb{R}}
%\newcommand{\ddd}{\mathrm{d}}
\newcommand{\rdd}{~\mathrm{d}}
\newcommand{\p}{\partial}
\newcommand{\st}{\mathrm{~s.t.~}}
\newcommand{\set}[1]{\left\{#1\right\}}

\begin{document}
\ioptwocol[{
\title{Iterations on the slimpletic integrator and its applications to physical systems}
\author{J D Coles$^1$}
\address{$^1$ Department of Physics, University of Bath, Claverton Down, Bath BA2 7AY, UK}
\begin{abstract}
\lipsum[1]
\end{abstract}
}]

\section{Introduction}
% Heres all the knowledge you need to understand shit

How do we talk about the SI before we talk about the maths?
\begin{enumerate}
	\item What is a slimpletic integrator
	\item What are they useful for
	\item What is the need to auto-diff and a higher speed one
	\item What \PHYSICS does this let us do?
	\item How do we get to autodiff
	\item Introduce the physical system which will be the through line of our model.
\end{enumerate}

\cite{tsangSLIMPLECTICINTEGRATORSVARIATIONAL2015}

% Slimpletic Integrator Maths
% Need to look at the papers before DT's

% Do we need to talk about the JAX things in the "understanding" component? Probably only briefly as its not **PHYSICS** sooooo
% Will need to talk about auto-diff. Link to JAX papers.
% Intro to gradient descent
	% Loss functions, what are they

\section{Method}
% What is the method for the JAX components
% I imagine learning from the DLA report the answer is... don't
% So what do we need to mention here? Probably the physical systems and methods used to measure performance, system size, etc.
% Then discuss auto-diff and the physical systems we chose to fit toward and why.
% How do the systems we fit towards relate to the through-line system?
% Different loss functions we chose to explore and why
% How we compare different loss functions

% What we did, how we built loss functios

\section{Results and Specific Discussion}
% This is the shit we got out of the method
% Essentially 
% These are the error graphs similar to DTs paper (cite DT here)
% We got X% speed up
% This means we can model Y% larger systems for Z% longer time frames
% Results of auto diff results with different loss functions
% If we can make those nice convergence/non-convergence Mandlebrot plots try that!

\lipsum

\section{General Discussion}
% What do the pretty graphs actually mean in a broad sense
% Pull together with the background
% PHYSICS!!!

% Is this where we will mention the "future work" of NNs?

\lipsum

\section{Conclusion}

\ack

\begin{enumerate}
	\item Tsang, resources
	\item Do we need to reference the Royal Society?
	\item Mathilda, Dunkun
\end{enumerate}

\section*{References}
\bibliographystyle{iopart-num}
\bibliography{references}

\end{document}
