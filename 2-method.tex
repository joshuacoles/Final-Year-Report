\section{Method}
%\subsection{Example Systems}
%\label{sec:eg-sys}
%
%When testing our model we employ a number of example systems to test various features of the integrator, in different physical scenarios. Each of these is also associated with a specific embedding.

%\subsubsection{Damped Forced Harmonic Oscillators}
\subsection{Damped Forced Harmonic Oscillators}
\label{sec:eg-sys}

The main physical system used in the testing, and evaluation, of our model is the damped forced harmonic oscillator. It is one of the simplest non-conservative systems, having a non-conservative Lagrangian given by \cite{galleyPrincipleStationaryNonconservative2014},

\begin{equation}
\label{eq:dho-sys-embedding}
\begin{aligned}
  \Lambda &= L - K \\
  L &= \frac12 m\dot\vbq^2 - \frac12 m\dot\vbq -  \\
  K &= - \lambda\vbq_- \cdot \dot \vbq_+ + \vbq_- \cdot \vb{F}(t)
\end{aligned}
\end{equation}

\noindent for a system with mass $m$, spring constant $k$, damping constant $\lambda$ and time dependent force $\vb{F}(t)$. This corresponds with the standard equation of motion,

\begin{equation}
  m \ddot{\vb x} + \lambda \dot{\vb x} + k \vb x = \vb F(t).
\end{equation}

\subsection{Improvements to the \SI{} and their Physical Applications}

Initially, the existing \texttt{slimplectic} codebase from Tsang \etall \cite{originalCode}, which we will refer to as the \orgimpl{}, was rewritten using the JAX framework, known as the \updimpl{}. We chose two key metrics, run-time duration and accuracy, to judge our model against the \orgimpl{} and Runge-Kutta to various orders.

First, to ensure that we maintained the error bounds expected for our model (as per \sref{sec:intro-si}) we compared the fractional energy and momentum error for Runge-Kutta, the \orgimpl{}, and the \updimpl{} across the time evolution of a physical system.

Next, we create two test cases to measure the performance of the integrator and its consequences on experimental capability. These were,

\begin{enumerate}
	\item Modelling systems for different timespans varying the number of iterations performed by the integrator.
	\item Modelling the same system across different orders of integration $r$ to determine how the order of our method affects the performance.
\end{enumerate}

These are designed to test the scaling behaviours of our model to provide insight into its applicability to larger systems.

\subsection{Applications to Loss Functions and Optimisation}

Taking this \updimpl{} we then explore its different uses as a loss function for physical systems, in combination with different embedding functions.

While these are distinct components of the model, they are heavily linked, as ideally, we would define the physical components of our loss function primarily in terms of the trajectories, rather than the values of the embedding vector for the non-conservative Lagrangian itself.
Hence the embedding function and Slimplectic integrator must be chained together, before the physical loss component, to provide it with the trajectories.

We consider a number of choices for these two functions, comparing their behaviour near known minima, and looking for convexity and smoothness. In addition, we also empirically test the suitability of these spaces for optimisation, by directly performing gradient descent to the loss function + embedding composition attempting to converge to a known embedding.

\subsection{Loss Function in the Training of PINNs}

Finally, we employ a loss and embedding function combination to a, simplified, stand-in neural network, observing the correctness of the outputs.

This model was based on the long short-term memory (LSTM) layer \cite{hochreiterLongShortTermMemory1997} which is frequently used in time-series analysis, and thus chosen as a suitable basis for our model. This involves generating large datasets of different physical systems within the domain of the chosen embedding as will be discussed.
